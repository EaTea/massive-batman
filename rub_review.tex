\section{Rubric usage review} \label{rubReview}

To evaluate the seminar, I was provided with a rubric which addressed talk attributes such as
structure, content, delivery and preparedness of the speaker.\\
\\
I was particularly interested in three questions regarding the rubric
\begin{enumerate}
	\item how does the rubric focus and improve seminar assessment?
	\item how does the rubric constraint thinking?
	\item what does current literature say about using a rubric for evaluation?
\end{enumerate}

To address the first question, I found that the rubric improved my understanding of what I should be
looking for in a good speaker.
Obviously, this is according to another person's definitions and views on what a good speaker is.
However, most of the recommendations seemed sound enough that I could place my trust in its usage.\\
\\
In my opinion the attributes that we were asked to use in our assessment were relevant and useful.
They were well-described, in that there was
\begin{itemize}
	\item enough flexibility for an assessor to bend the rubric to suit a situation
	\item enough information to know which criteria assessed which aspect of a seminar
\end{itemize}

As a result, I felt that they were reasonable to assess with.
Furthermore, by listing each of the criteria we were assessing on, the rubric explicitly stated what
to look for in a speaker and what to evaluate on.\\
\\
Addressing the second question involves recognising the drawbacks of listing each of the criteria.
We could have missed a criterion, which would leave our rubric incomplete and less useful.
An example of this is that no attribute about ``technical difficulties" with slides is found
anywhere in our rubric.
Not having this attribute made it difficult for me to effectively assess the speaker on fumbling
during his speech.\\
\\
A further weakness is revealed in the equal weighting that each item of the rubric gets.
As an example, we note that the assessment item

\begin{quote}
	The ``take home message" is clearly identified
\end{quote}
is equally weighted with another assessment item
\begin{quote}
	(The speaker is) Dressed/groomed appropriately
\end{quote}

It is debatable that the importance of a key message and how easily an audience understands it is
equal to that of how well-dressed a speaker is.
I believe this highlights a major weakness of an evaluation using a rubric --- it may give equal
importance and weighting during assessment to criteria which do not seem to be as important.
Having said this, it relies on my opinion of what a good speaker and communicator does, and it might
be the case that literature suggests grooming has as big an impact as a key message.\\
\\
Finally, the surrounding literature notes that rubrics have positive effects during teaching and
assessment.
Reddy and Andrade note in \cite{reddy2010review} that rubrics seem to show positive effects upon
assessment.
However, they further noted that clarity and appropriateness of language were a concern.
This is related to the views that I outlined earlier regarding the criteria we assessed our speaker
with and how descriptive the rubric was.
