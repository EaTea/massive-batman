\section{Presentation Review} \label{semRev}

Identifying good and bad aspects of a presentation are the first steps toward attributing good and
bad habits for presenters.
I review a science communication seminar in order to elicit the positive and negative components of
the presentation.
Using existing literature, I will justify my claims about good or bad presenting techniques and
discuss them with reference to previous research.
After highlighting these, I will construct a ``feedback sandwich" technique \FIXME in order to summarise
improvements and feedback.\\
\\
The seminar that I attended was centred around \FIXME.
The speaker intended to present to an audience of \FIXME.
The actual audience was \FIXME.
I assessed the presentation using a rubric.
More can be found about my analysis of the rubric in Section \ref{rubReview}.
Overall, I scored this presentation as 72 out of 110.

\subsection{Positive aspects of presentation} \label{semRevPlus}

\begin{itemize}
	\item content was suitable for intended audience
	\item well-motivated
	\item talk was well-referenced
\end{itemize}

\subsection{Negative aspects of presentation} \label{semRevMinus}

\begin{itemize}
	\item speaker was late
	\item jargon appeared and confused audience
	\item it was difficult to parse take home message (KM)
\end{itemize}

\subsection{Sample Feedback Sandwich} \label{semRevSandwich}

\begin{itemize}
	\item positives (content, citing)
	\item negatives
	\item positives (well-motivated)
\end{itemize}
