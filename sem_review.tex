\section{Presentation Review} \label{semRev}

Identifying good and bad aspects of a presentation are the first steps toward attributing good and
bad habits for presenters.
I reviewed a science communication seminar in order to elicit the positive and negative components of
the presentation.
I will attempt to justify my claims about good or bad presenting techniques and
discuss them with reference to previous research.\\
\\
The seminar I attended was centred around improving science education at the primary school level,
and was delivered to a group of science communication practioners and researchers.
I assessed the presentation using a rubric.
More can be found about my analysis of the rubric in Section \ref{rubReview}.
Overall, I scored this presentation as 72 out of 110.

\subsection{Positive aspects of presentation} \label{semRevPlus}

There were four particularly good things that were done in my seminar.\\
\\
Firstly, the speaker was well-groomed and dressed.
He chose ``smart-casual" attire, including a business shirt and pants, but no tie or jacket.
I believe this reflected the expectations of his audience, and gave his presence during the talk a
professional feel.
Dunbar and Segrin, as well as Shao et. al present experimental evidence that supports my impressions
\cite{shao2004effects, dunbar2012clothing}.\\
\\
Secondly, the content was relevant for both the intended audience, as well as non-experts in the
field.
The speaker recognised that his audience were mainly academics and science communication
practitioners, and spoke about research material that engaged and was relevant to his audience.
However, by highlighting (through visual media) the importance of his research and its impact upon
science communication and education in Australia the speaker improved my comprehension and
recognition of the relevance of his work.
Pelletier and Sharp suggest that framing a message to be relevant to the audience improves
understanding and allows the speaker to effectively communicate with a wider audience \cite{pelletier2008persuasive}.\\
\\
Thirdly, the seminar allowed ample time for questions.
Furthermore, the speaker was amiable when asked to clarify or answer questions about his work.
He provided satisactory and useful answers that assisted audience comprehension.
I felt he managed the impressions he made upon the audience well, by being assertive, calm and
showing that he was well-versed in the field.
Ellis et. al suggest that assertive, non-aggressive mannerisms in particular are well-received when
answering a question \cite{ellis2002use}.\\
\\
Finally, the speaker had a complete, formal and consistent bibliography.
He cited relevant literature and displayed it after his talk.
Furthermore, he referred to the workds he referenced when he spoke.
Chaiken and Maheswaran have found that objective assessment was more likely to occur for a
well-referenced and credible source \cite{chaiken1994heuristic}.
I would claim that by displaying a broad understanding of his research field
and topic, he appeared more credible.
As a result, the audience could trust his work and claims and objectively assess and analyse his
work, which would in turn improve our understanding of his presentation.

\subsection{Negative aspects of presentation} \label{semRevMinus}

In addition to positive aspects of the presentation, there were some flaws with the presenter's
mannerisms.
I will briefly discuss three of them, and if possible link any relevant material from the field to
support each one.\\
\\
Firstly, the speaker arrived 30 minutes late and as a result was rushed to complete his presentation.
Furthermore, he had more content than could have perhaps been covered in 45 minutes (the allotted
length of his presentation).
The relevant literature suggests that the speaker himself was negatively affected by being late and
under time pressure \cite{DeDreu2003280,Ahituv:1998:ETP:1189478.1189487}.
Indeed, the speaker at times had trouble with technology and was forced to skip slides due to being
under time pressure.\\
\\
It is an interesting question of how {\em I} was affected by his lateness and how my perception of
his competence changed.
My own view is that it negatively affected my perception of the speaker and detracted from his
presentation due to having to wait for, and possibly resent the speaker for his punctuality (or lack
thereof).
This is indirectly supported by literature, such as \cite{weber2002author,carson1998toward}, which
suggest that timely feedback and services are perceptions of quality and ``goodness".
However, I could not find any literature that directly supports this claim, and further research
would be required to substantiate it.\\
\\
Secondly, the speaker employed jargon and acronyms when discussing his work.
These included ``PISA" (Programme for International Student Assessment) and ``TIMMS" (Trends in
International Mathematics and Science Study).
Even experts in the audience were confused by some of the jargon the speaker used and asked him to
clarify what the jargon meant.
This detracted from the effectiveness of his speech, and made it more difficult to understand why
his content was relevant or useful.
Although the speaker did eventually explain it, an audience member had to clarify that she
was confused before he did so.\\
\\
Finally, I felt that the speaker did not make his key message clear until halfway through the
seminar.
Once he explicitly outlined this key message, it was easy to understand and appreciate the direction
of his talk.
However, it took a substantial amount of time before I knew what his key message was and why it was
important.
I felt that being more explicit and changing the structure of his talk could have really helped to
make the key message more explicit and relevant to his entire audience.
