\section{What makes a good science speaker?} \label{discussion}

After analysing a science seminar and methods of presentation assessment, it is
worthwhile to now consider --- what makes a good science communicator?
I will now draw away from the specific issues I noticed during my seminar
evaluation and instead discuss a wider range of essential (perhaps foundational) attributes of a
good science communicator.
I will answer this question with three key characteristics that are essential for good science
communication.\\
\\
Firstly, I believe a good science communicator speaks with intent.
That is, they have decided upon a powerful key message which is relevant and clear to their
audience from (almost) the moment they begin speaking.
It engages audiences and improves their understanding and recall of how and why the topic that is
discussed is relevant to them.
\cite{olsen1990point} show how important this is by exploring when science communication fails ---
here the audience understands the {\em words} in the seminar, but not the main point (key message) of the seminar.
Furthermore, the audience's comprehension was demonstrably worse if they could not grasp the main
point of the seminar.\\
\\
Furthermore, stating the key message in a clear and relevant manner improves the power of a
presentation.
This is shown by \cite{rasmuson1987effects} and \cite{Mitchell2013471}, who demonstrate that a
powerful speech can improve understanding and recall.
Rasmuson discusses measurable gains to understanding by adding pauses to significant parts of a
presentation.
Mitchell and Ross note that speaking clearly and confidently can improve the emotional connection an
audience develops with a topic, thereby improving the relevance of the topic to them.\\
\\
The second characteristic a good science communicator should have is to avoid jargon in order to
reach and impact a wider audience.
Jargon is more than technical words --- it is the usage of antiquated or uncommon words that are
confusing and difficult to understand.
Indeed there should be a minimal amount of jargon --- enough to explain
the concepts within a talk, and then {\em only} that jargon should be employed.
A medical study showed that many doctors employed jargon whilst they gave recommendations, and that
jargon severely decreased patients' understanding of what a physician diagnosed or recommended
\cite{castro2007babel}.
This effect is amplified when giving a talk to a large audience, since multiple people will be
confused by the talk.
This would further detract from the usefulness and relevance of the talk as a result.\\
\\
A final attribute of good science communicators is confident body language.
There is no betrayal of any nervousness they might experience on stage.
I have given presentations where I did not have confidence, and this was translated through my body
language and ruined that presentation.
An example of how good body language ``makes-or-breaks" a presentation is the ``Dr. Fox Lecture"
\cite{naftulin1973doctor}, where a fake, vacuous talk was given with confidence and good body
language so that the audience was convinced that the speaker was an expert.
I believe that a science communicator who can utilise the lessons from Dr. Fox to improve their
communication and speaking skills is a much better and more effective communicator overall.
